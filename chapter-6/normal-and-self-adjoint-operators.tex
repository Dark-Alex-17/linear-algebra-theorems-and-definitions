\section{Normal and Self-Adjoint Operators}

\begin{lemma}
	\hfill\\
	Let $T$ be a linear operator on a finite-dimensional inner product space $V$. If $T$ has an eigenvector, then so does $T^*$.
\end{lemma}

\begin{theorem}[\textbf{Schur}]
	\hfill\\
	Let $T$ be a linear operator on a finite-dimensional inner product space $V$. Suppose that the characteristic polynomial of $T$ splits. Then there exists an orthonormal basis $\beta$ for $V$ such that the matrix $[T]_\beta$ is upper triangular.
\end{theorem}

\begin{definition}
	\hfill\\
	Let $V$ be an inner product space, and let $T$ be a linear operator on $V$. We say that $T$ is \textbf{normal} if $TT^* = T^*T$. An $n \times n$ real or complex matrix $A$ is \textbf{normal} if $AA^* = A^*A$.
\end{definition}

\begin{theorem}
	\hfill\\
	Let $V$ be an inner product space, and let $T$ be a normal operator on $V$. Then the following statements are true.

	\begin{enumerate}
		\item $||T(x)|| = ||T^*(x)||$ for all $x \in V$.
		\item $T - cI$ is normal for every $c \in \F$.
		\item If $x$ is an eigenvector of $T$, then $x$ is also an eigenvector of $T^*$. In fact, if $T(x) = \lambda x$, then $T^*(x) = \overline{\lambda}x$.
		\item If $\lambda_1$ and $\lambda_2$ are distinct eigenvalues of $T$ with corresponding eigenvectors $x_1$ and $x_2$, then $x_1$ and $x_2$ are orthogonal.
	\end{enumerate}
\end{theorem}

\begin{theorem}
	\hfill\\
	Let $T$ be a linear operator on a finite-dimensional complex inner product space $V$. Then $T$ is normal if and only if there exists an orthonormal basis for $V$ consisting of eigenvectors of $T$.
\end{theorem}

\begin{definition}
	\hfill\\
	Let $T$ be a linear operator on an inner product space $V$. We say that $T$ is \textbf{self-adjoint} (\textbf{Hermitian}) if $T = T^*$. An $n \times n$ real or complex matrix $A$ is \textbf{self-adjoint} (\textbf{Hermitian}) if $A = A^*$.
\end{definition}

\begin{lemma}
	\hfill\\
	Let $T$ be a self-adjoint operator on a finite-dimensional inner product space $V$. Then

	\begin{enumerate}
		\item Every eigenvalue of $T$ is real.
		\item Suppose that $V$ is a real inner product space. Then the characteristic polynomial of $T$ splits.
	\end{enumerate}
\end{lemma}

\begin{theorem}
	\hfill\\
	Let $T$ be a linear operator on a finite-dimensional real inner product space $V$. Then $T$ is self-adjoint if and only if there exists an orthonormal basis $\beta$ for $V$ consisting of eigenvectors of $T$.
\end{theorem}

\begin{definition}
	\hfill\\
	An $n \times n$ real matrix $A$ is said to be a \textbf{Gramian} matrix if there exists a real (square) matrix $B$ such that $A = B^tB$.
\end{definition}

\begin{definition}
	\hfill\\
	A linear operator $T$ on a finite-dimensional inner product space is called \textbf{positive definite [positive semidefinite]} if $T$ is self-adjoint and $\lr{T(x),x} > 0$ [$\lr{T(x),x} \geq 0$] for all $x \neq 0$.

	An $n \times n$ matrix $A$ with entries from $\R$ or $\C$ is called \textbf{positive definite [positive semidefinite]} if $L_A$ is positive definite [positive semidefinite].
\end{definition}
