\section{Orthogonal Projections and the Spectral Theorem}


\begin{definition}
	\hfill\\
	Let $V$ be an inner product space, and let $T: V \to V$ be a projection. We say that $T$ is an \textbf{orthogonal projection} if $\range{T}^\perp = \n{T}$ and $\n{T}^\perp = \range{T}$.
\end{definition}

\begin{definition}
	\hfill\\
	Let $W$ be a finite-dimensional subspace of an inner product space $V$, and let $T$ be an orthogonal projection on $W$. We call $T$ the \textbf{orthogonal projection of $V$ on $W$}.
\end{definition}

\begin{definition}
	A \textbf{trigonometric polynomial of degree $n$} is a function $g \in \mathsf{H}$ of the form

	\[g(t) = \sum_{j=-n}^{n}a_jf_j(t) = \sum_{j=-n}^{n}a_je^{ijt},\]

	where $a_n$ or $a_{-n}$ is nonzero.
\end{definition}

\begin{theorem}
	\hfill\\
	Let $V$ be an inner product space, and let $T$ be a linear operator on $V$. Then $T$ is an orthogonal projection if and only if $T$ has an adjoint $T^*$ and $T^2 = T = T^*$.
\end{theorem}

\begin{theorem}[\textbf{The Spectral Theorem}]\label{The Spectral Theorem}
	\hfill\\
	Suppose that $T$ is a linear operator on a finite-dimensional inner product space $V$ over $\F$ with the distinct eigenvalues $\lambda_1, \lambda_2, \dots, \lambda_k$. Assume that $T$ is normal if $\F = \C$ and that $T$ is self-adjoint if $\F = \R$. For each $i$ ($1 \leq i \leq k$), let $W_i$ be the eigenspace of $T$ corresponding to the eigenvalue $\lambda_i$, and let $T_i$ be the orthogonal projection of $V$ on $W_i$. Then the following statements are true.

	\begin{enumerate}
		\item $V = W_1 \oplus W_2 \oplus \dots \oplus W_k$.
		\item If $W_i'$ denotes the direction sum of the subspaces $W_j$ for $j \neq i$, then $W_i^\perp = W_i'$.
		\item $T_iT_j = \delta_{ij}T_i$ for $1 \leq i, j \leq k$.
		\item $I = T_1 + T_2 + \dots + T_k$.
		\item $T = \lambda_1T_1 + \lambda_2T_2 + \dots + \lambda_kT_k$.
	\end{enumerate}
\end{theorem}

\begin{definition}
	\hfill\\
	In the context of \autoref{The Spectral Theorem}:

	\begin{enumerate}
		\item The set $\{\lambda_1, \lambda_2, \dots, \lambda_k\}$ of eigenvalues of $T$ is called the \textbf{spectrum} of $T$.
		\item The sum $I = T_1 + T_2 + \dots + T_k$ in (4) is called the \textbf{resolution of the identity operator} induced by $T$.
		\item The sum $T = \lambda_1T_1 + \lambda_2T_2 + \dots + \lambda_kT_k$ in (5) is called the \textbf{spectral decomposition} of $T$. The spectral decomposition of $T$ is unique up to the order of its eigenvalues.
	\end{enumerate}
\end{definition}

\begin{corollary}
	\hfill\\
	If $\F = \C$, then $T$ is normal if and only if $T^* = g(T)$ for some polynomial $g$.
\end{corollary}

\begin{corollary}
	\hfill\\
	If $\F = \C$, then $T$ is unitary if and only if $T$ is normal and $|\lambda| = 1$ for every eigenvalue of $T$.
\end{corollary}

\begin{corollary}
	\hfill\\
	If $\F = \C$ and $T$ is normal, then $T$ is self-adjoint if and only if every eigenvalue of $T$ is real.
\end{corollary}

\begin{corollary}
	\hfill\\
	Let $T$ be as in \autoref{The Spectral Theorem} with spectral decomposition $T = \lambda_1T_1 + \lambda_2T_2 + \dots + \lambda_kT_k$. Then each $T_j$ is a polynomial in $T$.
\end{corollary}
