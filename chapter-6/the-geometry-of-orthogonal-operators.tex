\section{The Geometry of Orthogonal Operators}

\begin{definition}
	\hfill\\
	Let $T$ be a linear operator on a finite-dimensional real inner product space $V$. The operator $T$ is called a \textbf{rotation} if $T$ is the identity on $V$ or if there exists a two-dimensional subspace $W$ of $V$, and orthonormal basis $\beta = \{x_1, x_2\}$ for $W$, and a real number $\theta$ such that

	\[T(x_1) = (\cos(\theta))x_1 + (\sin(\theta))x_2,\ \ \ \ T(x_2) = (-\sin(\theta))x_1 + (\cos(\theta))x_2,\]

	and $T(y) = y$ for all $y \in W^\perp$. In this context, $T$ is called a \textbf{rotation of $W$ about $W^\perp$}. The subspace $W^\perp$ is called the \textbf{axis of rotation}.
\end{definition}

\begin{definition}
	\hfill\\
	Let $T$ be a linear operator on a finite-dimensional real inner product space $V$. The operator $T$ is called a \textbf{reflection} if there exists a one-dimensional subspace $W$ of $V$ such that $T(x) = -x$ for all $x \in W$ and $T(y) = y$ for all $y \in W^\perp$. In this context, $T$ is called a \textbf{reflection of $V$ about $W^\perp$}.
\end{definition}

\begin{theorem}
	\hfill\\
	Let $T$ be an orthogonal operator on a two-dimensional real inner product space $V$. Then $T$ is either a rotation or a reflection. Furthermore, $T$ is a rotation if and only if $\det(T) = 1$, and $T$ is a reflection if and only if $\det(T) = -1$.
\end{theorem}

\begin{corollary}
	\hfill\\
	Let $V$ be a two-dimensional real inner product space. The composite of a reflection and a rotation on $V$ is a reflection on $V$.
\end{corollary}

\begin{lemma}
	\hfill\\
	If $T$ is a linear operator on a nonzero finite-dimensional real vector space $V$, then there exists a $T$-invariant subspace $W$ of $V$ such that $1 \leq \ldim{W} \leq 2$.
\end{lemma}

\begin{theorem}\label{Theorem 6.46}
	\hfill\\
	Let $T$ be an orthogonal operator on a nonzero finite-dimensional real inner product space $V$. Then there exists a collection of pairwise orthogonal $T$-invariant subspaces $\{W_1, W_2, \dots, W_m\}$ of $V$ such that

	\begin{enumerate}
		\item $1 \leq \ldim(W_i) \leq 2$ for $i = 1, 2, \dots, m$.
		\item $V = W_1 \oplus W_2 \oplus \dots \oplus W_m$.
	\end{enumerate}
\end{theorem}

\begin{theorem}
	\hfill\\
	Let $T,V,W_1,\dots,W_m$ be as in \autoref{Theorem 6.46}.

	\begin{enumerate}
		\item The number of $W_i$'s for which $T_{W_i}$ is a reflection is even or odd according to whether $\det(T) = 1$ or $\det(T) = -1$.
		\item It is always possible to decompose $V$ as in \autoref{Theorem 6.46} so that the number of $W_i$'s for which $T_{W_i}$ is a reflection is zero or one according to whether $\det(T) = 1$ or $\det(T) = -1$. Furthermore, if $T_{W_i}$ is a reflection, then $\ldim{W_i} = 1$.
	\end{enumerate}
\end{theorem}

\begin{corollary}
	\hfill\\
	Let $T$ be an orthogonal operator on a finite-dimensional real inner product space $V$. Then there exists a collection $\{T_1, T_2, \dots, T_m\}$ of orthogonal operators on $V$ such that the following statements are true.

	\begin{enumerate}
		\item For each $i$, $T_i$ is either a reflection or a rotation.
		\item For at most one $i$, $T_i$ is a reflection.
		\item $T_iT_j = T_jT_i$ for all $i$ and $j$.
		\item $T = T_1T_2\dots T_m$.
		\item $\det(T) = \displaystyle\begin{cases}
				      1  & \text{if}\ T_i\ \text{is a rotation for each}\ i \\
				      -1 & \text{otherwise}
			      \end{cases}$
	\end{enumerate}
\end{corollary}
