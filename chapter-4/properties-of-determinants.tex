\section{Properties of Determinants}

\begin{remark}
	\hfill\\
	Because the determinant of the $n \times n$ matrix is $1$, we can interpret \autoref{Remark 4.1} as the following facts about the determinants of elementary matrices.

	\begin{enumerate}
		\item If $E$ is an elementary matrix obtained by interchanging any two rows of $I$, then $\det(E) = -1$.
		\item If $E$ is an elementary matrix obtained by multiplying some row of $I$ by the nonzero scalar $k$, then $\det(E) = k$.
		\item If $E$ is an elementary matrix obtained by adding a multiple of some row of $I$ to another row, then $\det(E) = 1$.
	\end{enumerate}
\end{remark}

\begin{theorem}
	\hfill\\
	For any $A, B \in M_{n \times n}(\F)$, $\det(AB) = \det(A) \cdot \det(B)$.
\end{theorem}

\begin{corollary}
	\hfill\\
	A matrix $A \in M_{n \times n}(\F)$ is invertible if and only if $\det(A) \neq 0$. Furthermore, if $A$ is invertible, then $\det(A^{-1}) = \displaystyle\frac{1}{\det(A)}$.
\end{corollary}

\begin{theorem}
	\hfill\\
	For any $A \in M_{n \times n}(\F)$, $\det(A^t)=\det(A)$.
\end{theorem}

\begin{theorem}[\textbf{Cramer's Rule}]
	\hfill\\
	Let $Ax = b$ be the matrix form of a system of $n$ linear equations in $n$ unknowns, where $x = (x_1, x_2, \dots, x_n)^t$. If $\det(A) \neq 0$, then this system has a unique solution, and for each $k$ ($k = 1, 2, \dots, n$),

	\[x_k = \frac{\det(M_k)}{\det(A)},\]

	where $M_k$ is the $n \times n$ matrix obtained from $A$ by replacing column $k$ of $A$ by $b$.
\end{theorem}

\begin{definition}
	\hfill\\
	It is possible to interpret the determinant of a matrix $A \in M_{n \times n}(\R)$ geometrically. If the rows of $A$ are $a_1, a_2, \dots, a_n$, respectively, then $|\det(A)|$ is the \textbf{\textit{n}-dimensional volume} (the generalization of are in $\R^2$ and volume in $\R^3$) of the parallelepiped having the vectors $a_1, a_2, \dots, a_n$ as adjacent sides.
\end{definition}

\begin{definition}
	\hfill\\
	A matrix $M \in M_{n \times n}(\C)$ is called \textbf{nilpotent} if, for some positive integer $k$, $M^k = O$, where $O$ is the $n \times n$ zero matrix.
\end{definition}

\begin{definition}
	\hfill\\
	A matrix $M \in M_{n \times n}(\C)$ is called \textbf{skew-symmetric} if $M^t = -M$.
\end{definition}

\begin{definition}
	\hfill\\
	A matrix $Q \in M_{n \times n}(\R)$ is called \textbf{orthogonal} if $QQ^t = I$.
\end{definition}

\begin{definition}
	\hfill\\
	A matrix $Q \in M_{n \times n}(\C)$ is called \textbf{unitary} if $QQ^* = I$, where $Q^* = \overline{Q^t}$.
\end{definition}

\begin{definition}
	\hfill\\
	A matrix $A \in M_{n \times n}(\F)$ is called \textbf{lower triangular} if $A_{ij}=0$ for $1 \leq i < j \leq n$.
\end{definition}

\begin{definition}
	\hfill\\
	A matrix of the form

	\[\begin{pmatrix}
			1      & c_0    & c_0^2  & \dots & c_0^n  \\
			1      & c_1    & c_1^2  & \dots & c_1^n  \\
			\vdots & \vdots & \vdots &       & \vdots \\
			1      & c_n    & c_n^2  & \dots & c_n^n
		\end{pmatrix}\]

	is called a \textbf{Vandermonde matrix}.
\end{definition}

\begin{definition}
	\hfill\\
	Let $A \in M_{n \times n}(\F)$ be nonzero. For any $m$ ($1 \leq m \leq n$), and $m \times m$ \textbf{submatrix} is obtained by deleting any $n - m$ rows and any $n - m$ columns of $A$.
\end{definition}

\begin{definition}
	\hfill\\
	The \textbf{classical adjoint} of a square matrix $A$ is the transpose of the matrix whose $ij$-entry is the $ij$-cofactor of $A$.
\end{definition}

\begin{definition}
	\hfill\\
	Let $y_1, y_2, \dots, y_n$ be linearly independent function in $\C^\infty$. For each $y \in \C^\infty$, define $T(y) \in \C^\infty$ by

	\[[T(y)](t) = \det\begin{pmatrix}
			y(t)       & y_1(t)       & y_2(t)       & \dots & y_n(t)       \\
			y'(t)      & y'_1(t)      & y'_2(t)      & \dots & y'_n(t)      \\
			\vdots     & \vdots       & \vdots       &       & \vdots       \\
			y^{(n)}(t) & y_1^{(n)}(t) & y_2^{(n)}(t) & \dots & y_n^{(n)}(t)
		\end{pmatrix}\]

	The preceding determinant is called the \textbf{Wronskian} of $y, y_1, \dots, y_n$.
\end{definition}
