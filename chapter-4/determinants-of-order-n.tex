\section{Determinants of Order \textit{n}}

\begin{notation}
	\hfill\\
	Given $A \in M_{n \times n}(\F)$, for $n \geq 2$, denote the $(n-1) \times (n - 1)$ matrix obtained from $A$ by deleting row $i$ and column $j$ by $\tilde{A}_{ij}$. Thus for
	
	\[A = \begin{pmatrix}
		1 & 2 & 3 \\
		4 & 5 & 6 \\
		7 & 8 & 9
	\end{pmatrix} \in M_{3 \times 3}(\R)\]
	
	we have
	
	\[\tilde{A}_{11} = \begin{pmatrix}
			5 & 6 \\
			8 & 9
		\end{pmatrix},\ \ \ \ \ 
		\tilde{A}_{13}=\begin{pmatrix}
			4 & 5 \\
			7 & 8
		\end{pmatrix},\ \ \ \ \ 
		\tilde{A}_{32} = \begin{pmatrix}
			1 & 3 \\
			4 & 6
		\end{pmatrix}\]
		
	and for
	
	\[B = \begin{pmatrix}
		1 & -1 & 2 & -1 \\
		-3 & 4 & 1 & -1 \\
		2 & -5 & -3 & 8 \\
		-2 & 6 & -4 & 1
	\end{pmatrix}\]
	
	we have
	
	\[\tilde{B}_{23} = \begin{pmatrix}
		1 & -1 & -1 \\
		2 & -5 & 8 \\
		-2 & 6 & 1
	\end{pmatrix}\ \ \ \ \ \text{and}\ \ \ \ \ \tilde{B}_{42}=\begin{pmatrix}
		1 & 2 & -1 \\
		-3 & 1 & -1 \\
		2 & -3 & 8
	\end{pmatrix}\]
\end{notation}

\begin{definition}
	\hfill\\
	Let $A \in M_{n \times n}(\F)$. If $n =1$, so that $A = (A_{11})$, we define $\det(A) = A_{11}$. For $n \geq 2$, we define $\det(A)$ recursively as
	
	\[\det(A) = \sum_{j=1}^{n}(-1)^{1+j}A_{1j}\cdot\det(\tilde{A}_{1j}).\]
	
	The scalar $\det(A)$ is called the \textbf{determinant} of $A$ and is also denoted by $|A|$. The scalar
	
	\[(-1)^{i+j}\det(\tilde{A}_{ij})\]
	
	is called the \textbf{cofactor} of the entry of $A$ in row $i$, column $j$.
\end{definition}

\begin{definition}
	\hfill\\
	Letting
	
	\[c_{ij} = (-1)^{i+j}\det(\tilde{A}_{ij})\]
	
	denote the cofactor of the row $i$, column $j$ entry of $A$, we can express the formula for the determinant of $A$ as
	
	\[\det(A) = A_{11}c_{11} + A_{12}c_{12}+\dots+A_{1n}c_{1n}.\]
	
	Thus the determinant of $A$ equals the sum of the products of each entry in row $1$ of $A$ multiplied by its cofactor. This formula is called \textbf{cofactor expansion along the first row} of $A$.
\end{definition}

\begin{theorem}
	\hfill\\
	the determinant of an $n \times n$ matrix is a linear function of each row when the remaining rows are held fixed. That is, for $1 \leq r \leq n$, we have
	
	\[\det\begin{pmatrix}
		a_1 \\ \vdots \\ a_{r-1} \\ u+kv \\ a_{r+1} \\ \vdots \\ a_n
	\end{pmatrix}=\det\begin{pmatrix}
	a_1 \\ \vdots \\ a_{r-1} \\ u \\ a_{r+1} \\ \vdots \\ a_n
	\end{pmatrix} + k\det\begin{pmatrix}
	a_1 \\ \vdots \\ a_{r-1} \\ v \\ a_{r+1} \\ \vdots \\ a_n
	\end{pmatrix}\]
	
	wherever $k$ is a scalar and $u, v$ and each $a_i$ are row vectors in $\F^n$.
\end{theorem}

\begin{corollary}
	\hfill\\
	If $A \in M_{n \times n}(\F)$ has a row consisting entirely of zeros, then $\det(A) = 0$.
\end{corollary}

\begin{lemma}
	\hfill\\
	Let $B \in M_{n \times n}(\F)$, where $n \geq 2$. If row $i$ of $B$ equals $e_k$ for some $k$ ($1 \leq k \leq n$), then $\det(B) = (-1)^{i+k}\det(\tilde{B}_{ik})$.
\end{lemma}

\begin{theorem}
	\hfill\\
	The determinant of a square matrix can be evaluated by cofactor expansion along any row. That is, if $A \in M_{n \times n}(\F)$, then for any integer $i$ ($1 \leq i \leq n$),
	
	\[\det(A) = \sum_{j=1}^{n}(-1)^{i+j}A_{ij}\cdot\det(\tilde{A}_{ij}).\]
\end{theorem}

\begin{corollary}
	\hfill\\
	If $A \in M_{n \times n}(\F)$ has two identical rows, then $\det(A) = 0$.
\end{corollary}

\begin{theorem}
	\hfill\\
	If $A \in M_{n \times n}(\F)$ and $B$ is a matrix obtained from $A$ by interchanging any two rows of $A$, then $\det(B) = -\det(A)$.
\end{theorem}

\begin{theorem}
	\hfill\\
	Let $A \in M_{n \times n}(\F)$, and let $B$ be a matrix obtained by adding a multiple of one row of $A$ to another row of $A$. Then $\det(B) = \det(A)$.
\end{theorem}

\begin{corollary}
	\hfill\\
	If $A \in M_{n \times n}(\F)$ has rank less than $n$, then $\det(A) = 0$.
\end{corollary}

\begin{remark}\label{Remark 4.1}
	\hfill\\
	The following rules summarize the effect of an elementary row operation on the determinant of a matrix $A \ in M_{n \times n}(\F)$.
	
	\begin{enumerate}
		\item If $B$ is a matrix obtained by interchanging any two rows of $A$, then $\det(B) = -\det(A)$.
		\item If $B$ is a matrix obtained by multiplying a row of $A$ by a nonzero scalar $k$, then $\det(B) = k\det(A)$.
		\item If $B$ is a matrix obtained by adding a multiple of one row of $A$ to another row of $A$, then $\det(B) = \det(A)$.
	\end{enumerate}
\end{remark}

\begin{lemma}
	\hfill\\
	The determinant of an upper triangular matrix is the product of its diagonal entries.
\end{lemma}