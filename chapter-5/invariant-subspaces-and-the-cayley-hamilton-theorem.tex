\section{Invariant Subspaces and the Cayley-Hamilton Theorem}

\begin{definition}
	\hfill\\
	Let $T$ be a linear operator on a vector space $V$. A subspace $W$ of $V$ is called a \textbf{$T$-invariant subspace} of $V$ if $T(W) \subseteq W$, that is, if $T(v) \in W$ for all $v \in W$.
\end{definition}

\begin{definition}
	\hfill\\
	Let $T$ be a linear operator on a vector space $V$, and let $x$ be a nonzero vector in $V$. The subspace

	\[W = \lspan{\{x, T(x), T^2(x), \dots\}}\]

	is called the \textbf{$T$-cyclic subspace of $V$ generated by $x$}.
\end{definition}

\begin{theorem}
	\hfill\\
	Let $T$ be a linear operator on a finite-dimensional vector space $V$, and let $W$ be a $T$-invariant subspace of $V$. Then the characteristic polynomial of $T_W$ divides the characteristic polynomial of $T$.
\end{theorem}

\begin{theorem}\label{Theorem 5.22}
	\hfill\\
	Let $T$ be a linear operator on a finite-dimensional vector space $V$, and let $W$ denote the $T$-cyclic subspace of $V$ generated by a nonzero vector $v \in V$. Let $k = \ldim(W)$. Then

	\begin{enumerate}
		\item $\{v, T(v), T^2(v), \dots, T^{k-1}(v)\}$ is a basis for $W$.
		\item If $a_0v + a_1T(v) + \dots + a_{k-1}T^{k-1}(v)+T^k(v) = 0$, then the characteristic polynomial of $T_W$ is $f(t) = (-1)^k(a_o + a_1t + \dots +a_{k-1}t^{k-1}+t^k)$.
	\end{enumerate}
\end{theorem}

\begin{theorem}[\textbf{Cayley-Hamilton}]
	\hfill\\
	Let $T$ be a linear operator on a finite-dimensional vector space $V$, and let $f(t)$ be the characteristic polynomial of $T$. Then $f(T) = T_0$, the zero transformation. That is, $T$ ``satisfies" its characteristic equation.
\end{theorem}

\begin{corollary}[\textbf{Cayley-Hamilton Theorem for Matrices}]
	\hfill\\
	Let $A$ be an $n \times n$ matrix, and let $f(t)$ be the characteristic polynomial of $A$. Then $f(A) = O$, the $n \times n$ zero matrix.
\end{corollary}

\begin{theorem}
	\hfill\\
	Let $T$ be a linear operator on a finite-dimensional vector space $V$, and suppose that $V = W_1 \oplus W_2 \oplus \dots \oplus W_k$, where $W_i$ is a $T$-invariant subspace of $V$ for each $i$ ($1 \leq i \leq k$). Suppose that $f_i(t)$ is the characteristic polynomial of $T_{W_i}$ ($1 \leq i \leq k$). Then $f_1(t)\cdot f_2(t) \cdot \dots \cdot f_k(t)$ is the characteristic polynomial of $T$.
\end{theorem}

\begin{definition}
	\hfill\\
	Let $B_1 \in M_{m \times m}(\F)$, and let $B_2 \in M_{n \times n}(\F)$. We define the \textbf{direct sum} of $B_1$ and $B_2$, denoted $B_1 \oplus B_2$, as the $(m + n) \times (m + n)$ matrix $A$ such that

	\[A_{ij} = \begin{cases}
			(B_1)_{ij}          & \text{for}\ 1 \leq i, j \leq m         \\
			(B_2)_{(i-m),(j-m)} & \text{for}\ m + 1 \leq i, j \leq n + m \\
			0                   & \text{otherwise.}
		\end{cases}\]

	If $B_1, B_2, \dots, B_k$ are square matrices with entries from $\F$, then we define the \textbf{direct sum} of $B_1, B_2, \dots, B_k$ recursively by

	\[B_1 \oplus B_2 \oplus \dots \oplus B_k = (B_1 \oplus B_2 \oplus \dots \oplus B_{k-1})\oplus B_k.\]

	If $A= B_1 \oplus B_2 \oplus \dots \oplus B_k$, then we often write

	\[A = \begin{pmatrix}
			B_1    & O      & \dots & O      \\
			O      & B_2    & \dots & O      \\
			\vdots & \vdots &       & \vdots \\
			O      & O      & \dots & B_k
		\end{pmatrix}\]
\end{definition}

\begin{theorem}
	\hfill\\
	Let $T$ be a linear operator on a finite-dimensional vector space $V$, and let $W_1, W_2, \dots, W_k$ be $T$-invariant subspaces of $V$ such that $V = W_1 \oplus W_2 \oplus \dots \oplus W_k$. For each $i$, let $\beta_i$ be an ordered basis for $W_i$, and let $\beta = \beta_1 \cup \beta_2 \cup \dots \cup \beta_k$. Let $A = [T]_\beta$ and $B_i = [T_{W_i}]_{\beta_i}$ for $i = 1, 2, \dots, k$. Then $A = B_1 \oplus B_2 \oplus \dots \oplus B_k$.
\end{theorem}

\begin{definition}
	\hfill\\
	Let $T$ be a linear operator on a vector space $V$, and let $W$ be a $T$-invariant subspace of $V$. Define $\overline{T}: V/W \to V/W$ by

	\[\overline{T}(v + W) = T(v) + W\ \ \ \text{for any}\ v + W \in V/W.\]
\end{definition}
