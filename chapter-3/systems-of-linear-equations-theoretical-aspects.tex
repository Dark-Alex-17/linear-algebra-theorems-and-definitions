\section{Systems of Linear Equations -- Theoretical Aspects}

\begin{definition}
	\hfill\\
	The system of equations

	\begin{equation}\label{eq:S}
		\tag{S}
		\begin{split}
			a_{11}x_1 + a_{12}x_2 + \dots + a_{1n}x_n = b_1\\
			a_{21}x_1 + a_{22}x_2 + \dots + a_{2n}x_n = b_2\\
			\dots \\
			a_{m1}x_1 + a_{m2}x_2 + \dots + a_{mn}x_n = b_m,
		\end{split}
	\end{equation}

	where $a_{ij}$ and $b_i$ ($1 \leq i \leq m$ and $1 \leq j \leq n$) are scalars in a field $\F$ and $x_1, x_2, \dots, x_n$ are $n$ variables taking values in $\F$, is a called a \textbf{system of $m$ linear equations in $n$ unknowns over the field $\F$}.

	The $m \times n$ matrix

	\[\begin{pmatrix}
			a_{11} & a_{12} & \dots & a_{1n} \\
			a_{21} & a_{22} & \dots & a_{2n} \\
			\vdots & \vdots &       & \vdots \\
			a_{m1} & a_{m2} & \dots & a_{mn}
		\end{pmatrix}\]

	is called the \textbf{coefficient matrix} of the system \eqref{eq:S}.

	If we let

	\[x = \begin{pmatrix}
			x_1 \\ x_2 \\ \vdots \\ x_n
		\end{pmatrix}\ \ \text{and}\ \ b = \begin{pmatrix}
			b_1 \\ b_2 \\ \vdots \\ b_m
		\end{pmatrix},\]

	then the system \eqref{eq:S} may be rewritten as a single matrix equation

	\[Ax = b.\]

	To exploit the results that we have developed, we often consider a system of linear equations as a single matrix equation.

	A \textbf{solution} to the system \eqref{eq:S} is an $n$-tuple

	\[s = \begin{pmatrix}
			s_1 \\ s_2 \\ \vdots \\ s_n
		\end{pmatrix} \in \F^n\]

	such that $As = b$. The set of all solutions to the system \eqref{eq:S} is called the \textbf{solution set} of the system. System \eqref{eq:S} is called \textbf{consistent} if its solution set is nonempty; otherwise it is called \textbf{inconsistent}.
\end{definition}

\begin{definition}
	\hfill\\
	A system $Ax = b$ of $m$ linear equations in $n$ unknowns is said to be \textbf{homogeneous} if $b = 0$. Otherwise the system is said to be \textbf{nonhomogeneous}.\\

	Any homogeneous system has at least one solution, namely, the zero vector.
\end{definition}

\begin{theorem}
	\hfill\\
	Let $Ax = 0$ be a homogeneous system of $m$ linear equations in $n$ unknowns over a field $\F$. Let $K$ denote the set of all solutions to $Ax = 0$. Then $K = \n{L_A}$; hence $K$ is a subspace of $\F^n$ of dimension $n - \rank{L_A} = n - \rank{A}$.
\end{theorem}

\begin{corollary}
	\hfill\\
	If $m < n$, the system $Ax = 0$ has a nonzero solution.
\end{corollary}

\begin{definition}
	\hfill\\
	We refer to the equation $Ax = 0$ as the \textbf{homogeneous system corresponding to $Ax = b$}.
\end{definition}

\begin{theorem}
	\hfill\\
	Let $K$ be the solution set of a system of linear equations $Ax = b$, and let $\mathsf{K}_\mathsf{H}$ be the solution set of the corresponding homogeneous system $Ax = 0$. Then for any solution $s$ to $Ax = b$

	\[K = \{s\} + \mathsf{K}_\mathsf{H} = \{s + k: k \in \mathsf{K}_\mathsf{H}\}.\]
\end{theorem}

\begin{theorem}
	\hfill\\
	Let $Ax = b$ be a system of $n$ linear equations in $n$ unknowns. If $A$ is invertible, then the system has exactly one solution, namely, $A^{-1}b$. Conversely, if the system has exactly one solution, then $A$ is invertible.
\end{theorem}

\begin{definition}
	\hfill\\
	The matrix $(A|b)$ is called the \textbf{augmented matrix of the system $Ax = b$}.
\end{definition}

\begin{theorem}
	\hfill\\
	Let $Ax = b$ be a system of linear equations. Then the system is consistent if and only if $\rank{A} = \rank{A|b}$.
\end{theorem}

\subsection*{An Application}
\addcontentsline{toc}{subsection}{An Application}

\begin{definition}
	Consider a system of linear equations

	\[\begin{split}
			a_{11}p_1 + a_{12}p_2 + \dots + a_{1m}p_m = p_1 \\
			a_{21}p_1 + a_{22}p_2 + \dots + a_{2m}p_m = p_2 \\
			\dots \\
			a_{n1}p_1 + a_{n2}p_2 + \dots + a_{nm}p_m = p_m \\
		\end{split}\]

	This system can be written as $Ap = p$, where

	\[p = \begin{pmatrix}
			p_1 \\ p_2 \\ \vdots \\ p_m
		\end{pmatrix}\]

	and $A$ is the coefficient matrix of the system. In this context, $A$ is called the \textbf{input-output (or consumption) matrix}, and $Ap = p$ is called the \textbf{equilibrium condition}.

	For vectors $b = (b_1, b_2, \dots, b_n)$ and $c = (c_1, c_2, \dots, c_n)$ in $\R^n$, we use the notation $b \geq c$ [$b > c$] to mean $b_i \geq c_i$ [$b_i > c_i$] for all $i$. The vector $b$ is called \textbf{non-negative [positive]} if $b \geq 0$ [$b > 0$].
\end{definition}

\begin{theorem}
	\hfill\\
	Let $A$ be an $n \times n$ input-output matrix having the form

	\[A = \begin{pmatrix}
			B & C \\
			D & E
		\end{pmatrix},\]

	where $D$ is a $1 \times (n -1)$ positive vector and $C$ is an $(n-1)\times 1$ positive vector. Then $(I -A)x = 0$ has a one-dimensional solution set that is generated by a non-negative vector.
\end{theorem}
