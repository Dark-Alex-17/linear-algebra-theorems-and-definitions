\section{Systems of Linear Equations -- Computational Aspects}

\begin{definition}
	\hfill\\
	Two systems of linear equations are called \textbf{equivalent} if they have the same solution set.
\end{definition}

\begin{theorem}
	\hfill\\
	Let $Ax = b$ be a system of $m$ linear equations in $n$ unknowns, and let $C$ be an invertible $m \times n$ matrix. Then the system $(CA)x = Cb$ is equivalent to $Ax = b$.
\end{theorem}a

\begin{corollary}
	\hfill\\
	Let $Ax = b$ be a system of $m$ linear equations in $n$ unknowns. If $(A'|b')$ is obtained from $(A|b)$ by a finite number of elementary row operations, then the system $A'x = b'$ is equivalent to the original system.
\end{corollary}

\begin{definition}
	\hfill\\
	A matrix is said to be in \textbf{reduced row echelon form} if the following three conditions are satisfied.
	
	\begin{enumerate}
		\item Any row containing a nonzero entry precedes any row in which all the entries are zero (if any).
		
		\item The first nonzero entry in each row is the only nonzero entry in its column.
		
		\item The first nonzero entry in each row is 1 and it occurs in a column to the right of the first nonzero entry in the preceding row.
	\end{enumerate}
\end{definition}

\begin{definition}
	\hfill\\
	The following procedure for reducing an augmented matrix to reduced row echelon form is called \textbf{Gaussian elimination}. It consists of two separate parts.
	
	\begin{enumerate}
		\item In the \textit{forward pass}, the augmented matrix is transformed into an upper triangular matrix in which the first nonzero entry of each row is $1$, and it occurs in a column to the right of the first nonzero entry in the preceding row.
		
		\item In the \textit{backward pass} or \textit{back-substitution}, the upper triangular matrix is transformed into reduced row echelon form by making the first nonzero entry of each row the only nonzero entry of its column.
	\end{enumerate}
\end{definition}

\begin{theorem}
	\hfill\\
	Gaussian elimination transforms any matrix into its reduced row echelon form.
\end{theorem}

\begin{definition}
	A solution to a system of equations of the form
	
	\[s = s_0 + t_1u_1 + t_2u_2 + \dots +t_{n-r}u_{n-r},\]
	
	where $r$ is the number of nonzero solutions in $A'$ ($r \leq m$), is called a \textbf{general solution} of the system $Ax = b$. It expresses an arbitrary solution $s$ of $Ax = b$ in terms of $n - r$ parameters.
\end{definition}

\begin{theorem}
	\hfill\\
	Let $Ax = b$ be a system of $r$ nonzero equations in $n$ unknowns. Suppose that $\rank{A} = \rank{A|b}$ and that $(A|b)$ is in reduced row echelon form. Then
	
	\begin{enumerate}
		\item $\rank{A} = r$.
		\item If the general solution obtained by the procedure above is of the form
		
		\[s = s_0 + t_1u_1 + t_2u_2 + \dots + t_{n-r}u_{n-r},\]
		
		then $\{u_1, u_2, \dots, u_{n-r}\}$ is a basis for the solution set of the corresponding homogeneous system, and $s_0$ is a solution to the original system.
	\end{enumerate}
\end{theorem}

\begin{theorem}
	\hfill\\
	Let $A$ be an $m \times n$ matrix of rank $r$, where $r > 0$, and let $B$ be the reduced row echelon form of $A$. Then
	
	\begin{enumerate}
		\item The number of nonzero rows in $B$ is $r$.
		\item For each $i = 1, 2, \dots, r$, there is a column $b_{j_i}$ of $B$ such that $b_{j_i} = e_i$.
		\item The columns of $A$ numbered $j_1, j_2, \dots, j_r$ are linearly independent.
		\item For each $k = 1, 2, \dots, n$, if column $k$ of $B$ is $d_1e_1+d_2e_2+\dots+d_re_r$, then column $k$ of $A$ is $d_1a_{j_1} + d_2a_{j_2} + \dots + d_ra_{j_r}$.
	\end{enumerate}
\end{theorem}

\begin{corollary}
	\hfill\\
	The reduced row echelon form of a matrix is unique.
\end{corollary}