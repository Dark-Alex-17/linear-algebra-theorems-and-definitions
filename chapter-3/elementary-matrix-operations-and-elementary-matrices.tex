\section{Elementary Matrix Operations and Elementary Matrices}

\begin{definition}
	\hfill\\
	Let $A$ be an $m \times n$ matrix. Any one of the following three operations on the rows [columns] of $A$ is called an \textbf{elementary row [column] operation}:

	\begin{enumerate}
		\item interchanging any two rows [columns] of $A$;
		\item multiplying any row [column] of $A$ by a nonzero scalar;
		\item adding any scalar multiple of a row [column] of $A$ to another row [column].
	\end{enumerate}

	Any of these three operations are called an \textbf{elementary operation}. Elementary operations are of \textbf{type 1}, \textbf{type 2}, or \textbf{type 3} depending on whether they are obtained by (1), (2), or (3).
\end{definition}

\begin{definition}
	\hfill\\
	An $n \times n$ \textbf{elementary matrix} is a matrix obtained by performing an elementary operation on $I_n$. The elementary matrix is said to be of \textbf{type 1}, \textbf{2}, or \textbf{3} according to whether the elementary operation performed on $I_n$ is a type 1, 2, or 3 operation, respectively.
\end{definition}

\begin{theorem}
	\hfill\\
	Let $A \in M_{m \times n}(\F)$, and suppose that $B$ is obtained from $A$ by performing an elementary row [column] operation. Then there exists an $m \times m$ [$n \times n$] elementary matrix $E$ such that $B = EA$ [$B = AE]$. In fact, $E$ is obtained from $I_m$ [$I_n]$ by performing the same elementary row [column] operation as that which was performed on $A$ to obtain $B$. Conversely, if $E$ is an elementary $m \times m$ [$n \times n$] matrix, then $EA$ [$AE$] is the matrix obtained from $A$ by performing the same elementary row [column] operation as that which produces $E$ from $I_m$ [$I_n$].
\end{theorem}

\begin{theorem}
	\hfill\\
	Elementary matrices are invertible, and the inverse of an elementary matrix is an elementary matrix of the same type.
\end{theorem}
