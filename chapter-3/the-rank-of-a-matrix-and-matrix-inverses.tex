\section{The Rank of a Matrix and Matrix Inverses}

\begin{definition}
	\hfill\\
	If $A \in M_{m \times n}(\F)$, we define the \textbf{rank} of $A$, denoted $\rank{A}$, to be the rank of the linear transformation $L_A: \F^n \to \F^m$.
\end{definition}

\begin{theorem}
	\hfill\\
	Let $T: V \to W$ be a linear transformation between finite-dimensional vector spaces, and let $\beta$ and $\gamma$ be ordered bases for $V$ and $W$, respectively. Then $\rank{T} = \rank{[T]_\beta^\gamma}$.
\end{theorem}

\begin{theorem}
	\hfill\\
	Let $A$ be an $m \times n$ matrix. if $P$ and $Q$ are invertible $m \times m$ and $n \times n$ matrices, respectively, then

	\begin{enumerate}
		\item $\rank{AQ} = \rank{A}$,
		\item $\rank{PA} = \rank{A}$,\\ and therefore
		\item $\rank{PAQ} = \rank{A}$.
	\end{enumerate}
\end{theorem}

\begin{corollary}
	\hfill\\
	Elementary row and column operations on a matrix are rank preserving.
\end{corollary}

\begin{theorem}
	\hfill\\
	The rank of any matrix equals the maximum number of its linearly independent columns; that is, the rank of a matrix is the dimension of the subspace generated by its columns.
\end{theorem}

\begin{theorem}
	\hfill\\
	Let $A$ be an $m \times n$ matrix of rank $r$. Then $r \leq m$, $r \leq n$, and, by means of a finite number of elementary row and column operations, $A$ can be transformed into the matrix

	\[D = \begin{pmatrix}
			I_r & O_1 \\
			O_2 & O_3
		\end{pmatrix}\]

	where $O_1$, $O_2$ and $O_3$ are the zero matrices. Thus $D_{ii} = 1$ for $i \leq r$ and $D_{ij} = 0$ otherwise.
\end{theorem}

\begin{corollary}
	\hfill\\
	Let $A$ be an $m \times n$ matrix of rank $r$. Then there exist invertible matrices $B$ and $C$ of sizes $m \times m$ and $n \times n$, respectively, such that $D=BAC$, where

	\[D = \begin{pmatrix}
			I_r & O_1 \\
			O_2 & O_3
		\end{pmatrix}\]
	is the $m \times n$ matrix in which $O_1$, $O_2$, and $O_3$ are zero matrices.
\end{corollary}

\begin{corollary}
	\hfill\\
	Let $A$ be an $m \times n$ matrix. Then

	\begin{enumerate}
		\item $\rank{A^t} = \rank{A}$.
		\item The rank of any matrix equals the maximum number of its linearly independent rows; that is, the rank of a matrix is the dimension of the subspace generated by its rows.
		\item The rows and columns of any matrix generate subspaces of the same dimension, numerically equal to the rank of the matrix.
	\end{enumerate}
\end{corollary}

\begin{corollary}
	\hfill\\
	Every invertible matrix is a product of elementary matrices.
\end{corollary}

\begin{theorem}
	\hfill\\
	Let $T: V \to W$ and $U: W \to Z$ be linear transformations on finite-dimensional vector spaces $V$, $W$, and $Z$, and let $A$ and $B$ be matrices such that the product $AB$ is defined. Then

	\begin{enumerate}
		\item $\rank{UT} \leq \rank{U}$.
		\item $\rank{UT} \leq \rank{T}$.
		\item $\rank{AB} \leq \rank{A}$.
		\item $\rank{AB} \leq \rank{B}$.
	\end{enumerate}
\end{theorem}

\begin{definition}
	\hfill\\
	Let $A$ and $B$ be $m \times n$ and $m \times p$ matrices, respectively. By the \textbf{augmented matrix} $(A|B)$, we mean the $m \times (n \times p)$ matrix $(A\ B)$, that is, the matrix whose first $n$ columns are the columns of $A$, and whose last $p$ columns are the columns of $B$.
\end{definition}
