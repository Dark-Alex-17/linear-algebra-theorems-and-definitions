\begin{alphasection}
	\setcounter{alphasect}{4}
	\section{Polynomials}

	\begin{definition}
		\hfill\\
		A polynomial $f(x)$ \textbf{divides} a polynomial $g(x)$ if there exists a polynomial $q(x)$ such that $g(x) = f(x)q(x)$.
	\end{definition}

	\begin{theorem}[\textbf{The Division Algorithm for Polynomials}]\label{Theorem 8.7}
		Let $f(x)$ be a polynomial of degree $n$, and let $g(x)$ be a polynomial of degree $m \geq 0$. Then there exists unique polynomials $q(x)$ and $r(x)$ such that

		\[f(x) = q(x)g(x)+r(x),\]

		where the degree of $r(x)$ is less than $x$.
	\end{theorem}

	\begin{definition}
		\hfill\\
		In the context of \autoref{Theorem 8.7}, we call $q(x)$ and $r(x)$ the \textbf{quotient} and \textbf{remainder}, respectively, for the division of $f(x)$ by $g(x)$.
	\end{definition}

	\begin{corollary}
		\hfill\\
		Let $f(x)$ be a polynomial of positive degree, and let $a \in \F$. Then $f(a) = 0$ if and only if $x-a$ divides $f(x)$.
	\end{corollary}

	\begin{definition}
		\hfill\\
		For any polynomial $f(x)$ with coefficients from a field $\F$, an element $a \in \F$ is called a \textbf{zero} of $f(x)$ if $f(a) =0$. With this terminology, the preceding corollary states that $a$ is a zero of $f(x)$ if and only if $x-a$ divides $f(x)$.
	\end{definition}

	\begin{corollary}
		\hfill\\
		Any polynomial of degree $n \geq 1$ has at most $n$ distinct zeros.
	\end{corollary}

	\begin{definition}
		\hfill\\
		Two nonzero polynomials are called \textbf{relatively prime} if no polynomial of positive degree divides each of them.
	\end{definition}

	\begin{theorem}
		\hfill\\
		If $f_1(x)$ and $f_2(x)$ are relatively prime polynomials, there exist polynomials $q_1(x)$ and $q_2(x)$ such that

		\[q_1(x)f_1(x) + q_2(x)f_2(x) = 1,\]

		where $1$ denoted the constant polynomial with value $1$.
	\end{theorem}

	\begin{definition}
		\hfill\\
		Let

		\[f(x) = a_0 + a_1(x) + \dots + a_nx^n\]

		be a polynomial with coefficients from a field $\F$. If $T$ is a linear operator on a vector space $V$ over $\F$, we define

		\[f(T) = a_0I + a_1T + \dots + a_nT^n.\]

		Similarly, if $A$ is an $n \times n$ matrix with entries from $\F$, we define

		\[f(A) = a_0I+ a_1A + \dots + a_nA^n.\]
	\end{definition}

	\begin{theorem}
		\hfill\\
		Let $f(x)$ be a polynomial with coefficients from a field $\F$, and let $T$ be a linear operator on a vector space $V$ over $\F$. Then the following statements are true.

		\begin{enumerate}
			\item $f(T)$ is a linear operator on $V$.
			\item If $\beta$ is a finite ordered basis for $V$ and $A=[T]_\beta$, then $[f(T)]_\beta = f(A)$.
		\end{enumerate}
	\end{theorem}

	\begin{theorem}
		\hfill\\
		Let $T$ be a linear operator on a vector space $V$ over a field $\F$, and let $A$ be a square matrix with entries from $\F$. Then, for any polynomials $f_1(x)$ and $f_2(x)$ with coefficients $\F$,

		\begin{enumerate}
			\item $f_1(T)f_2(T) = f_2(T)f_1(T)$
			\item $f_1(A)f_2(A) = f_2(A)f_1(A)$.
		\end{enumerate}
	\end{theorem}

	\begin{theorem}
		\hfill\\
		Let $T$ be a linear operator on a vector space $V$ over a field $\F$, and let $A$ be an $n \times n$ matrix with entries from $\F$. If $f_1(x)$ and $f_2(x)$ are relatively prime polynomials with entries from $\F$, then there exist polynomials $q_1(x)$ and $q_2(x)$ with entries from $\F$ such that

		\begin{enumerate}
			\item $q_1(T)f_1(T) + q_2(T)f_2(T) = I$
			\item $q_1(A)f_1(A) + q_2(A)f_2(A) = I$.
		\end{enumerate}
	\end{theorem}

	\begin{definition}
		\hfill\\
		A polynomial $f(x)$ with coefficients from a field $\F$ is called \textbf{monic} if its leading coefficient is 1. If $f(x)$ has positive degree and cannot be expressed as a product of polynomials with coefficients from $\F$ each having positive degree, then $f(x)$ is called \textbf{irreducible}.
	\end{definition}

	\begin{theorem}
		\hfill\\
		Let $\phi(x)$ and $f(x)$ be polynomials. If $\phi(x)$ is irreducible and $\phi(x)$ does not divide $f(x)$, then $\phi(x)$ and $f(x)$ are relatively prime.
	\end{theorem}

	\begin{theorem}
		\hfill\\
		Any two distinct irreducible monic polynomials are relatively prime.
	\end{theorem}

	\begin{theorem}
		\hfill\\
		Let $f(x)$, $g(x)$, and $\phi(x)$ be polynomials. If $\phi(x)$ is irreducible and divides the product $f(x)g(x)$, then $\phi(x)$ divides $f(x)$ or $\phi(x)$ divides $g(x)$.
	\end{theorem}

	\begin{corollary}
		\hfill\\
		Let $\phi(x),\phi_1(x)\phi_2(x), \dots, \phi_n(x)$ be irreducible monic polynomials. If $\phi(x)$ divides the product $\phi_1(x) \phi_2(x) \dots \phi_n(x)$, then $\phi(x) = \phi_i(x)$ for some $i$ ($i = 1, 2, \dots n$).
	\end{corollary}

	\begin{theorem}[\textbf{Unique Factorization Theorem for Polynomials}]
		\hfil\\
		For any polynomial $f(x)$ of positive degree, there exist a unique constant $c$; unique distinct irreducible monic polynomials $\phi_1(x),\phi_2(x), \dots, \phi_n(x)$; and unique positive integers $n_1, n_2, \dots, n_k$ such that

		\[f(x) = c[\phi_1(x)]^{n_1} [\phi_2(x)]^{n_2} \dots [\phi_k(x)]^{n_k}.\]
	\end{theorem}

	\begin{theorem}
		\hfill\\
		Let $f(x)$ and $g(x)$ be polynomials with coefficients from an infinite field $\F$. If $f(a)= g(a)$ for all $a \in \F$, then $f(x)$ and $g(x)$ are equal.
	\end{theorem}
\end{alphasection}
