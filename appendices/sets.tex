\begin{alphasection}
	\section{Sets}

	\begin{definition}
		\hfill\\
		A \textbf{set} is a collection of objects, called \textbf{elements} of the set. If $x$ is an element of the set $A$, then we write $x \in A$; otherwise, we write $x \notin A$.
	\end{definition}

	\begin{definition}
		\hfill\\
		Two sets $A$ and $B$ are called \textbf{equal}, written $A = B$, if they contain exactly the same elements.
	\end{definition}

	\begin{remark}
		\hfill\\
		Sets may be described in one of two ways

		\begin{enumerate}
			\item By listing the elements of the set between set braces $\{$ $\}$.
			\item By describing the elements of the set in terms of some characteristic property.
		\end{enumerate}
	\end{remark}

	\begin{definition}
		\hfill\\
		A set $B$ is called a \textbf{subset} of $A$, written $B \supseteq A$ or $A \supseteq B$, if every element of $B$ is an element of $A$. If $B \subseteq A$, and $B \neq A$, then $B$ is called a \textbf{proper subset} of $A$.
	\end{definition}

	\begin{definition}
		\hfill\\
		The \textbf{empty set}, denoted by $\emptyset$, is the set containing no elements. The empty set is a subset of every set.
	\end{definition}

	\begin{definition}
		\hfill\\
		The \textbf{union} of two sets $A$ and $B$, denoted $A \cup B$, is the set of elements that are in $A$, or $B$, or both; that is,

		\[A \cup B = \{x : x \in A\ \text{or}\ B\}.\]
	\end{definition}

	\begin{definition}
		\hfill\\
		The \textbf{intersection} of two sets $A$ and $B$, denoted $A \cap B$, is the set of elements that are in both $A$ and $B$; that is,

		\[A \cap B = \{x : x \in A\ \text{and}\ B\}.\]
	\end{definition}

	\begin{definition}
		\hfill\\
		Two sets are called \textbf{disjoint} if their intersection equals the empty set.
	\end{definition}

	\begin{definition}
		\hfill\\
		A \textbf{relation} on a set $A$ is a set $S$ of ordered pairs of elements of $A$ such that $(x,y) \in S$ if and only if $x$ stands in the given relationship to $y$.
	\end{definition}

	\begin{definition}
		\hfill\\
		A relation $S$ on a set $A$ is called an \textbf{equivalence relation} on $A$ if the following three conditions hold:

		\begin{enumerate}
			\item For each $x \in A$, $x \sim x$  (reflexivity).
			\item If $x \sim y$, then $y \sim x$  (symmetry).
			\item If $x \sim y$ and $y \sim z$, then $x \sim z$  (transitivity).
		\end{enumerate}
	\end{definition}
\end{alphasection}
