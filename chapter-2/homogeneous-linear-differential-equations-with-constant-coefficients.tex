\section{Homogeneous Linear Differential Equations with Constant Coefficients}

\begin{definition}
	\hfill\\
	A \textbf{differential equation} in an unknown function $y = y(t)$ is an equation involving $y$, $t$, and derivatives of $y$. If the differential equation is of the form
	
	\begin{equation}
		a_ny^{(n)}+a_{n-1}y^{(n-)} + \dots + a_1y^{(1)}+a_0y = f,
	\end{equation}
	
	where $a_0, a_1, \dots, a_n$ and $f$ are functions of $t$ and $y^{(k)}$ denotes the $k$th derivative of $y$, then the equation is said to be \textbf{linear}. The functions $a_i$ are called the \textbf{coefficients} of the differential equation. When $f$ is identically zero, (2.1) is called \textbf{homogeneous}.\\
	
	If $a_n \neq 0$, we say that differential equation (2.1) is of \textbf{order \textit{n}}. In this case, we divide both sides by $a_n$ to obtain a new, but equivalent, equation
	
	\[y^{(n)} + b_{n-1}y^{(n-1)} + \dots + b_1y^{(1)} + b_0y = 0,\]
		
	where $b_i = a_i/a_n$ for $i=0, 1, \dots, n-1$. Because of this observation, we always assume that the coefficient $a_n$ in (2.1) is $1$.\\
	
	A \textbf{solution} to (2.1) is a function that when substituted for $y$ reduces (2.1) to an identity.
\end{definition}

\begin{definition}
	\hfill\\
	Given a complex-valued function $x \in \mathcal{F}(\R, \C)$ of a real variable $t$ (where $\mathcal{F}(\R, \C)$ is the vector space defined in \autoref{Definition 1.7}), there exist unique real-valued functions $x_1$ and $x_2$ of $t$, such that
	
	\[x(t) = x_1(t) + ix_2(t)\ \ \ \text{for}\ \ \ t \in \R,\]
	
	where $i$ is the imaginary number such that $i^2 = -1$. We call $x_1$ the \textbf{real part} and $x_2$ the \textbf{imaginary part} of $x$.
\end{definition}

\begin{definition}
	\hfill\\
	Given a function $x \in \mathcal{F}(\R, \C)$ with real part $x_1$ and imaginary part $x_2$, we say that $x$ is \textbf{differentiable} if $x_1$ and $x_2$ are differentiable. If $x$ is differentiable, we define the \textbf{derivative} $x'$ of $x$ by
	
	\[x' = x'_1 + ix'_2\]
\end{definition}

\begin{theorem}
	\hfill\\
	Any solution to a homogeneous linear differential equation with constant coefficients has derivatives of all orders; that is, if $x$ is a solution to such an equation, then $x^(k)$ exists for every positive integer $k$.
\end{theorem}

\begin{definition}
	\hfill\\
	We use $\C^\infty$ to denote the set of all functions in $\mathcal{F}(\R, \C)$ that have derivatives of all orders.
\end{definition}

\begin{definition}
	\hfill\\
	For any polynomial $p(t)$ over $\C$ of positive degree, $p(D)$ is called a \textbf{differential operator}. The \textbf{order} of the differential operator $p(D)$ is the degree of the polynomial $p(t)$.
\end{definition}

\begin{definition}
	\hfill\\
	Given the differential equation
	
	\[y^{(n)} + a_{n-1}y^{(n-1)}+ \dots + a_1y^{(1)} + a_0y = 0,\]
	
	the complex polynomial 
	
	\[p(t) = t^n + a_{n-1}t^{n-1} + \dots + a_1t + a_0\]
	
	is called the \textbf{auxiliary polynomial} associated with the equation.
\end{definition}

\begin{theorem}
	\hfill\\
	The set of all solutions to a homogeneous linear differential equation with constant coefficients coincides with the null space of $p(D)$ where $p(t)$ is the auxiliary polynomial associated with the equation.
\end{theorem}

\begin{corollary}
	\hfill\\
	The set of all solutions to a homogeneous linear differential equation with constant coefficients is a subspace of $\C^\infty$.
\end{corollary}

\begin{definition}
	\hfill\\
	We call the set of solutions to a homogeneous linear differential equation with constant coefficients the \textbf{solution space} of the equation.
\end{definition}

\begin{definition}
	\hfill\\
	Let $c = a+ib$ be a complex number with real part $a$ and imaginary part $b$. Define
	
	\[e^c = e^a(\cos(b) + i\sin(b)).\]
	
	The special case
	
	\[e^{ib} = \cos(b) + i\sin(a)\]
	
	is called \textbf{Euler's formula}.
\end{definition}

\begin{definition}
	\hfill\\
	A function $f: \R \to \C$ defined by $f(t) = e^{ct}$ for a fixed complex number $c$ is called an \textbf{exponential function}.
\end{definition}

\begin{theorem}
	\hfill\\
	For any exponential function $f(t) = e^{ct}$, $f'(t) = ce^{ct}$.
\end{theorem}

\begin{theorem}
	\hfill\\
	Recall that the \textbf{order} of a homogeneous linear differential equation is the degree of its auxiliary polynomial. Thus, an equation of order 1 is of the form
	
	\begin{equation}
		y' + a_0y = 0.
	\end{equation}
	
	The solution space for (2.2) is of dimension 1 and has $\{e^{-a_0t}\}$ as a basis.
\end{theorem}

\begin{corollary}
	\hfill\\
	For any complex number $c$, the null space of the differential operator $D-c\mathsf{l}$ has $\{e^{ct}\}$ as a basis.
\end{corollary}

\begin{theorem}
	\hfill\\
	Let $p(t)$ be the auxiliary polynomial for a homogeneous linear differential equation with constant coefficients. For any complex number $c$, if $c$ is a zero of $p(t)$, then $e^{ct}$ is a solution to the differential equation.
\end{theorem}

\begin{theorem}
	\hfill\\
	For any differential operator $p(D)$ of order $n$, the null space of $p(D)$ is an $n$-dimensional subspace of $\C^\infty$.
\end{theorem}

\begin{lemma}
	\hfill\\
	The differential operator $D - c\mathsf{l}: \C^\infty \to \C^\infty$ is onto for any complex number $c$.
\end{lemma}

\begin{lemma}
	\hfill\\
	Let $V$ be a vector space, and suppose that $T$ and $U$ are linear operators on $V$ such that $U$ is onto and the null spaces of $T$ and $U$ are finite-dimensional. Then the null space of $TU$ is finite-dimensional, and
	
	\[\ldim{\n{TU}} = \ldim{\n{T}} + \ldim{\n{U}}\]
\end{lemma}

\begin{corollary}
	\hfill\\
	The solution space of any $n$th-order homogeneous linear differential equation with constant coefficients is an $n$-dimensional subspace of $\C^\infty$.
\end{corollary}

\begin{theorem}
	\hfill\\
	Given $n$ distinct complex numbers $c_1, c_2, \dots, c_n$, the set of exponential functions $\{e^{c_1t},e^{c_2t},\dots,e^{c_nt}\}$ is linearly independent.
\end{theorem}

\begin{corollary}
	\hfill\\
	For any $n$th-order homogeneous linear differential equation with constant coefficients, if the auxiliary polynomial has $n$ distinct zeros $c_1, c_2, \dots, c_n$, then $\{e^{c_1t}, e^{c_2t}, \dots, e^{c_nt}\}$ is a basis for the solution space of the differential equation.
\end{corollary}

\begin{lemma}
	\hfill\\
	For a given complex number $c$ and a positive integer $n$, suppose that $(t-c)^n$ is the auxiliary polynomial of a homogeneous linear differential equation with constant coefficients. Then the set
	
	\[\beta = \{e^{ct}, te^{ct}, \dots, t^{n-1}e^{ct}\}\]
	
	is a basis for the solution space of the equation.
\end{lemma}

\begin{theorem}
	\hfill\\
	Given a homogeneous linear differential equation with constant coefficients and auxiliary polynomial
	
	\[(t-c_1)^{n_1}(t-c_2)^{n_2}\dots(t-c_k)^{n_k},\]
	
	where $n_1, n_2, \dots, n_k$ are positive integers and $c_1, c_2, \dots, c_k$ are distinct complex numbers, the following set is a basis for the solution space of the equation:
	
	\[\{e^{c_1t}, te^{c_1t},\dots, t^{n_1-1}e^{c_1t}, \dots, e^{c_kt}, te^{c_kt}, \dots, t^{n_k-1}e^{c_kt}\}\]
\end{theorem}

\begin{definition}
	\hfill\\
	A differential equation
	
	\[y^{(n)} + a_{n-1}y^{(n-1)} + \dots + a_1y^{(1)} + a_0y = x\]
	
	is called a \textbf{nonhomogeneous} linear differential equation with constant coefficients if the $a_i$'s are constant and $x$ is a function that is not identically zero.
\end{definition}