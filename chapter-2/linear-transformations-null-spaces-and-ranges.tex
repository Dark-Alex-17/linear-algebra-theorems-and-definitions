\section{Linear Transformations, Null Spaces, and Ranges}

\begin{definition}
	\hfill\\
	Let $V$ and $W$ be vector spaces (over $\F$). We call a function $T: V \to W$ a \textbf{linear transformation from $V$ to $W$} if, for all $x,y \in V$, and $c \in \F$, we have
	
	\begin{enumerate}
		\item $T(x + y) = T(x) + T(y)$, and
		\item $T(cx) = cT(x)$
	\end{enumerate}
	
	If the underlying field $\F$ is the field of rational numbers, then (1) implies (2), but, in general (1) and (2) are logically independent.\\
	
	We often simply call $T$ \textbf{linear}.
\end{definition}

\begin{remark}
	\hfill\\
	Let $V$ and $W$ be vector spaces (over $\F$). Let $T: V \to W$ be a linear transformation. Then the following properties hold:
	
	\begin{enumerate}
		\item If $T$ is linear, then $T(0) = 0$.
		\item $T$ is linear if and only if $T(cx + y) = cT(x) + T(y)$ for all $x,y \in V$ and $c \in \F$.
		\item If $T$ is linear, then $T(x-y)=T(x)-T(y)$ for all $x,y \in V$.
		\item $T$ is linear if and only if, for $x_1, x_2, \dots, x_n \in V$ and $a_1, a_2, \dots, a_n \in \F$, we have
		
		\[T\left(\sum_{i=1}^{n}a_ix_i\right)=\sum_{i=1}^{n}a_iT(x_i).\]
	\end{enumerate}
	
	We generally use property 2 to prove that a given transformation is linear.
\end{remark}

\begin{definition}
	\hfill\\
	For any angle $\theta$, define $T_\theta: \R^2 \to \R^2$ by the rule: $T_\theta(a_1, a_2)$ is the vector obtained by rotating $(a_1, a_2)$ counterclockwise by $\theta$ if $(a_1, a_2) \neq (0, 0)$, and $T_\theta(0,0) = (0,0)$. Then $T_\theta: \R^2 \to \R^2$ is a linear transformation that is called the \textbf{rotation by $\theta$}.
\end{definition}

\begin{definition}
	\hfill\\
	Define $T: \R^2 \to \R^2$ by $T(a_1, a_2) = (a_1, -a_2)$. $T$ is called the \textbf{reflection about the \textit{x}-axis}.
\end{definition}

\begin{definition}
	\hfill\\
	For vector spaces $V$ and $W$ (over $\F$), we define the \textbf{identity transformation} $I_V: V \to V$ by $I_V(x) = x$ for all $x \in V$.\\
	
	We define the \textbf{zero transformation} $T_0: V \to W$ by $T_0(x) = 0$ for all $x \in V$.\\
	
	\textbf{Note:} We often write $I$ instead of $I_V$.
\end{definition}

\begin{definition}
	\hfill\\
	Let $V$ and $W$ be vector spaces, and let $T: V \to W$ be linear. We define the \textbf{null space} (or \textbf{kernel}) $\n{T}$ to be the set of all vectors $x \in V$ such that $T(x)=0$; that is, \\$\n{T} = \{x \in V\ |\ T(x) = 0\}$.
	
	We define the \textbf{range} (or \textbf{image}) $\range{T}$ of $T$ to be the subset of $W$ consisting of all images (under $T$) of vectors in $V$; that is, $\range{T} = \{T(x)\ |\ x \in V\}$.
\end{definition}

\begin{theorem}
	\hfill\\
	Let $V$ and $W$ be vector spaces and $T: V \to W$ be linear. Then $\n{T}$ and $\range{T}$ are subspaces of $V$ and $W$, respectively.
\end{theorem}

\begin{theorem}
	\hfill\\
	Let $V$ and $W$ be vector spaces, and let $T: V \to W$ be linear. If $\beta = \{v_1, v_2, \dots, v_n\}$ is a basis for $V$, then
	
	\[\range{T} = \lspan{T(\beta)} = \lspan{\{T(v_1), T(v_2), \dots, T(v_n)\}}.\]
\end{theorem}

\begin{definition}
	\hfill\\
	Let $V$ and $W$ be vector spaces, and let $T: V \to W$ be linear. If $\n{T}$ and $\range{T}$ are finite-dimensional, then we define the \textbf{nullity} of $T$, denoted $\nullity{T}$, and the \textbf{rank} of $T$, denoted $\rank{T}$, to be the dimensions of $\n{T}$ and $\range{T}$, respectively.
\end{definition}

\begin{theorem}[\textbf{Dimension Theorem}]
	\hfill\\
	Let $V$ and $W$ be vector spaces, and let $T: V \to W$ be linear. If $V$ is finite-dimensional, then
	
	\[\nullity{T} + \rank{T} = \ldim{V}\]
\end{theorem}

\begin{theorem}
	\hfill\\
	Let $V$ and $W$ be vector spaces, and let $T: V \to W$ be linear. Then $T$ is one-to-one if and only if $\n{T} = \{0\}$.
\end{theorem}

\begin{theorem}\label{Theorem 2.5}
	\hfill\\
	Let $V$ and $W$ be vector spaces of equal (finite) dimension, and let $T: V \to W$ be linear. Then the following are equivalent.
	
	\begin{enumerate}
		\item $T$ is one-to-one.
		\item $T$ is onto.
		\item $\rank{T} = \ldim{V}$.
	\end{enumerate}
\end{theorem}

\begin{theorem}
	\hfill\\
	Let $V$ and $W$ be vector spaces over $\F$, and suppose that $\{v_1, v_2, \dots, v_n\}$ is a basis for $V$. For $w_1, w_2, \dots, w_n$ in $W$, there exists exactly one linear transformation $T: V \to W$ such that $T(v_i) = w_i$ for $i = 1, 2, \dots, n$.
\end{theorem}

\begin{corollary}\label{Corollary 2.1}
	\hfill\\
	Let $V$ and $W$ be vector spaces, and suppose that $V$ has a finite basis $\{v_1, v_2, \dots, v_n\}$. If $U,T: V \to W$ are linear and $U(v_i) = T(v_i)$, for $i = 1, 2, \dots, n$, then $U = T$.
\end{corollary}

\begin{definition}
	\hfill\\
	Let $V$ be a vector space and $W_1$ and $W_2$ be subspaces of $V$ such that $V = W_1 \oplus W_2$. A function $T: V \to V$ is called the \textbf{projection on $W_1$ along $W_2$} if, for $x = x_1 + x_2$ with $x_1 \in W$ and $x_2 \in W_2$, we have $T(x) = x_1$.
\end{definition}

\begin{definition}
	\hfill\\
	Let $V$ be a vector space, and let $T: V \to W$ be linear. A subspace $W$ of $V$ is said to be \textbf{$T$-invariant} if $T(x) \in W$ for every $x \in W$, that is, $T(W) \subseteq W$. If $W$ is $T$-invariant, we define the \textbf{restriction of $T$ on $W$} to be the function $T_W: W \to W$ defined by $T_W(x) = T(x)$ for all $x \in W$.
\end{definition}