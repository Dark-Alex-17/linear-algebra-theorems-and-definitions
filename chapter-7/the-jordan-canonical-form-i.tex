\section{The Jordan Canonical Form I}

\begin{definition}
	In this section, we extend the definition of eigenspace to \textit{generalized eigenspace}. From these subspaces, we select ordered bases whose union is an ordered basis $\beta$ for $V$ such that

	\[[T]_\beta = \begin{pmatrix}
			A_1    & O      & \dots & O      \\
			O      & A_2    & \dots & O      \\
			\vdots & \vdots &       & \vdots \\
			O      & O      & \dots & A_k
		\end{pmatrix}\]

	where each $O$ is a zero matrix, and each $A_i$ is a square matrix of the form $(\lambda$) or

	\[\begin{pmatrix}
			\lambda & 1       & 0      & \dots & 0       & 0       \\
			0       & \lambda & 1      & \dots & 0       & 0       \\
			\vdots  & \vdots  & \vdots &       & \vdots  & \vdots  \\
			0       & 0       & 0      & \dots & \lambda & 1       \\
			0       & 0       & 0      & \dots & 0       & \lambda
		\end{pmatrix}\]

	for some eigenvalue $\lambda$ of $T$. Such a matrix $A_i$ is called a \textbf{Jordan block} corresponding to $\lambda$, and the matrix $[T]_\beta$ is called a \textbf{Jordan canonical form} of $T$. We also say that the ordered basis $\beta$ is a \textbf{Jordan canonical basis} for $T$. Observe that each Jordan block $A_i$ is ``almost" a diagonal matrix -- in fact, $[T]_\beta$ is a diagonal matrix if and only if each $A_i$ is of the form $(\lambda)$.
\end{definition}

\begin{definition}
	\hfill\\
	Let $T$ be a linear operator on a vector space $V$, and let $\lambda$ be a scalar. A nonzero vector $x$ in $V$ is called a \textbf{generalized eigenvector of $T$ corresponding to $\lambda$} if $(T -\lambda I)^p(x) = 0$ for some positive integer $p$.
\end{definition}

\begin{definition}
	\hfill\\
	Let $T$ be a linear operator on a vector space $V$, and let $\lambda$ be an eigenvalue of $T$. The \textbf{generalized eigenspace of $T$ corresponding to $\lambda$}, denoted $K_\lambda$, is the subset of $V$ defined by

	\[K_\lambda = \{x \in V : (T - \lambda I)^p(x) = 0\ \text{for some positive integer}\ p\}.\]
\end{definition}

\begin{theorem}\label{Theorem 7.1}
	\hfill\\
	Let $T$ be a linear operator on a vector space $V$, and let $\lambda$ be an eigenvalue of $T$. then

	\begin{enumerate}
		\item $K_\lambda$ is a $T$-invariant subspace of $V$ containing $E_\lambda$ (the eigenspace of $T$ corresponding to $\lambda$).
		\item for any scalar $\mu \neq \lambda$, the restriction of $T - \mu I$ to $K_\lambda$ is one-to-one.
	\end{enumerate}
\end{theorem}

\begin{theorem}
	\hfill\\
	Let $T$ be a linear operator on a finite-dimensional vector space $V$ such that the characteristic polynomial of $T$ splits. Suppose that $\lambda$ is an eigenvalue of $T$ with multiplicity $m$. Then

	\begin{enumerate}
		\item $\ldim{K_\lambda} \leq m$.
		\item $K_\lambda = \n{(T - \lambda I)^m}$.
	\end{enumerate}
\end{theorem}

\begin{theorem}
	\hfill\\
	Let $T$ be a linear operator on a finite-dimensional vector space $V$ such that the characteristic polynomial of $T$ splits, and let $\lambda_1, \lambda_2, \dots, \lambda_k$ be the distinct eigenvalues of $T$. Then, for every $x \in V$, there exist vectors $v_i \in K_\lambda$, $1 \leq i \leq k$, such that

	\[x = v_1 + v_2 + \dots + v_k.\]
\end{theorem}

\begin{theorem}\label{Theorem 7.4}
	\hfill\\
	Let $T$ be a linear operator on a finite-dimensional vector space $V$ such that the characteristic polynomial of $T$ splits, and let $\lambda_1, \lambda_2, \dots, \lambda_k$ be the distinct eigenvalues of $T$ with corresponding multiplicities $m_1, m_2, \dots, m_k$. For $1 \leq i \leq k$, let $\beta_i$ be an ordered basis for $K_{\lambda_i}$. Then the following statements are true.

	\begin{enumerate}
		\item $\beta_i \cap \beta_j = \emptyset$ for $i \neq j$.
		\item $\beta = \beta_1 \cup \beta_2 \cup \dots \cup \beta_k$ is an ordered basis for $V$.
		\item $\ldim{K_{\lambda_i}} = m_i$ for all $i$.
	\end{enumerate}
\end{theorem}

\begin{corollary}
	\hfill\\
	Let $T$ be a linear operator on a finite-dimensional vector space $V$ such that the characteristic polynomial of $T$ splits. Then $T$ is diagonalizable if and only if $E_\lambda = K_\lambda$ for every eigenvalue $\lambda$ of $T$.
\end{corollary}

\begin{definition}
	\hfill\\
	Let $T$ be a linear operator on a vector space $V$, and let $x$ be a generalized eigenvector of $T$ corresponding to the eigenvalue $\lambda$. Suppose that $p$ is the smallest positive integer for which $(T - \lambda I)^p(x) = 0$. Then the ordered set

	\[\{(T-\lambda I)^{p-1}(x), (T-\lambda I)^{p -2}(x), \dots, (T-\lambda I)(x), x\}\]
	is called a \textbf{cycle of generalized eigenvectors} of $T$ corresponding to $\lambda$. The vectors $(T-\lambda I)^{p-1}(x)$ and $x$ are called the \textbf{initial vector} and the \textbf{end vector} of the cycle, respectively. We say that the \textbf{length} of the cycle is $p$.
\end{definition}

\begin{theorem}\label{Theorem 7.5}
	\hfill\\
	Let $T$ be a linear operator on a finite-dimensional vector space $V$ whose characteristic polynomial splits, and suppose that $\beta$ i a basis for $V$ such that $\beta$ is a disjoint union of cycles of generalized eigenvectors of $T$. Then the following statements are true.

	\begin{enumerate}
		\item For each cycle $\gamma$ of generalized eigenvectors contained in $\beta$, $W = \lspan{\gamma}$ is $T$-invariant, and $[T_W]_\gamma$ is a Jordan block.
		\item $\beta$ is a Jordan canonical basis for $V$.
	\end{enumerate}
\end{theorem}

\begin{theorem}
	\hfill\\
	Let $T$ be a linear operator on a vector space $V$, and let $\lambda$ be an eigenvalue of $T$. Suppose that $\gamma_1, \gamma_2, \dots,\gamma_q$ are cycles of generalized eigenvectors of $T$ corresponding to $\lambda$ such that the initial vectors of the $\gamma_i$'s are distinct and form a linearly independent set. Then the $\gamma_i$'s are disjoint, and their union $\gamma = \displaystyle\bigcup_{i = 1}^q \gamma_i$ is linearly independent.
\end{theorem}

\begin{corollary}
	\hfill\\
	Every cycle of generalized eigenvectors of a linear operator is linearly independent.
\end{corollary}

\begin{theorem}\label{Theorem 7.7}
	\hfill\\
	Let $T$ be a linear operator on a finite-dimensional vector space $V$, and let $\lambda$ be an eigenvalue of $T$. Then $K_\lambda$ has an ordered basis consisting of a union of disjoint cycles of generalized eigenvectors corresponding to $\lambda$.
\end{theorem}

\begin{corollary}
	\hfill\\
	Let $T$ be a linear operator on a finite-dimensional vector space $V$ whose characteristic polynomial splits. then $T$ has Jordan canonical form.
\end{corollary}

\begin{definition}
	\hfill\\
	Let $A \in M_{n \times n}(\F)$ be such that the characteristic polynomial of $A$ (and hence of $L_A$) splits. Then the \textbf{Jordan canonical form} of $A$ is defined to be the Jordan canonical form of the linear operator $L_A$ on $\F^n$.
\end{definition}

\begin{corollary}
	\hfill\\
	Let $A$ be an $n \times n$ matrix whose characteristic polynomial splits. Then $A$ has Jordan canonical form $J$, and $A$ is similar to $J$.
\end{corollary}

\begin{theorem}
	\hfill\\
	Let $T$ be a linear operator on a finite-dimensional vector space $V$ whose characteristic polynomial splits. Then $V$ is the direct sum of the generalized eigenspaces of $T$.
\end{theorem}
