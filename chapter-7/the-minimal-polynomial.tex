\section{The Minimal Polynomial}

\begin{definition}
	\hfill\\
	A polynomial $f(x)$ with coefficients from a field $\F$ is called \textbf{monic} if its leading coefficient is $1$. If $f(x)$ has positive degree and cannot be expressed as a product of polynomials with coefficients from $\F$ each having positive degree, then $f(x)$ is called \textbf{irreducible}.
\end{definition}

\begin{definition}
	Let $T$ be a linear operator on a finite-dimensional vector space. A polynomial $p(t)$ is called a \textbf{minimal polynomial} of $T$ if $p(t)$ is a monic polynomial of least positive degree for which $p(T) = T_0$.
\end{definition}

\begin{theorem}
	\hfill\\
	Let $p(t)$ be a minimal polynomial of a linear operator $T$ on a finite-dimensional vector space $V$.

	\begin{enumerate}
		\item For any polynomial $g(t)$, if $g(T) = T_0$, then $p(t)$ divides $g(t)$. In particular, $p(t)$ divides the characteristic polynomial of $T$.
		\item The minimal polynomial $T$ is unique.
	\end{enumerate}
\end{theorem}

\begin{definition}
	\hfill\\
	Let $A \in M_{n \times n}(\F)$. The \textbf{minimal polynomial} $p(t)$ of $A$ is the monic polynomial of least positive degree for which $p(A) = O$.
\end{definition}

\begin{theorem}
	\hfill\\
	Let $T$ be a linear operator on a finite-dimensional vector space $V$, and let $\beta$ be an ordered basis for $V$. Then the minimal polynomial of $T$ is the same as the minimal polynomial of $[T]_\beta$.
\end{theorem}

\begin{corollary}
	\hfill\\
	For any $A \in M_{n \times n}(\F)$, the minimal polynomial of $A$ is the same as the minimal polynomial of $L_A$.
\end{corollary}

\begin{theorem}
	\hfill\\
	Let $T$ be a linear operator on a finite-dimensional vector space $V$, and let $p(t)$ be the minimal polynomial of $T$. A scalar $\lambda$ is an eigenvalue of $T$ if and only if $p(\lambda) = 0$. Hence the characteristic polynomial and the minimal polynomial of $T$ have the same zeros.
\end{theorem}

\begin{corollary}
	\hfill\\
	Let $T$ be a linear operator on a finite-dimensional vector space $V$ with minimal polynomial $p(t)$ and characteristic polynomial $f(T)$. Suppose that $f(t)$ factors as

	\[f(t) = (\lambda_1 - f)^{n_1}(\lambda_2 - t)^{n_2} \dots(\lambda_k - t)^{n_k},\]

	where $\lambda_1, \lambda_2, \dots, \lambda_k$ are the distinct eigenvalues of $T$. Then there exist integers $m_1, m_2, \dots, m_k$ such that $1 \leq m_i \leq n_i$ for all $i$ and

	\[p(t) = (t - \lambda_1)^{m_1}(t - \lambda_2)^{m_2}\dots(t - \lambda_k)^{m_k}.\]
\end{corollary}

\begin{theorem}\label{Theorem 7.15}
	\hfill\\
	Let $T$ be a linear operator on an $n$-dimensional vector space $V$ such that $V$ is a $T$-cyclic subspace of itself. Then the characteristic polynomial $f(t)$ and the minimal polynomial $p(t)$ have the same degree, and hence $f(t) = (-1)^np(t)$.
\end{theorem}

\begin{theorem}
	\hfill\\
	Let $T$ be a linear operator on a finite-dimensional vector space $V$. Then $T$ is diagonalizable if and only if the minimal polynomial of $T$ is of the form

	\[p(t) = (t - \lambda_1)(t - \lambda_2) \dots (t - \lambda_k),\]

	where $\lambda_1, \lambda_2, \dots, \lambda_k$ are the distinct eigenvalues of $T$.
\end{theorem}

\begin{definition}
	\hfill\\
	Let $T$ be a linear operator on a finite-dimensional vector space $V$, and let $x$ be a nonzero vector in $V$. The polynomial $p(t)$ is called a $T$-\textbf{annihilator} of $x$ if $p(t)$ is a monic polynomial of lest degree for which $p(T)(x) = 0$.
\end{definition}
