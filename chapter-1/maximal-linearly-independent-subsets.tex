\section{Maximal Linearly Independent Subsets}

\begin{definition}
	\hfill\\
	Let $\mathcal{F}$ be a family of sets. A member $M$ of $\mathcal{F}$ is called \textbf{maximal} (with respect to set inclusion) if $M$ is contained in no member of $\mathcal{F}$ other than $M$ itself.
\end{definition}

\begin{definition}
	\hfill\\
	Let $\mathcal{F}$ be the family of all subsets of a nonempty set $S$. This family $\mathcal{F}$ is called the \textbf{power set} of $S$. 
\end{definition}

\begin{definition}
	\hfill\\
	A collection of sets $\mathcal{C}$ is called a \textbf{chain} (or \textbf{nest} or \textbf{tower}) if for each pair of sets $A$ and $B$ in $\mathcal{C}$, either $A \subseteq B$ or $B \subseteq A$.
\end{definition}

\begin{definition}[\textbf{Maximal Principle}]
	\hfill\\
	Let $\mathcal{F}$ be a family of sets. If, for each chain $\mathcal{C} \subseteq \mathcal{F}$, there exists a member of $\mathcal{F}$ that contains each member of $\mathcal{C}$, then $\mathcal{F}$ contains a maximal member.\\
	
	\textbf{Note:} The \textit{Maximal Principle} is logically equivalent to the \textit{Axiom of Choice}, which is an assumption in most axiomatic developments of set theory.
\end{definition}

\begin{definition}
	\hfill\\
	Let $S$ be a subset of a vector space $V$. A \textbf{maximal linearly independent subset} of $S$ is a subset $B$ of $S$ satisfying both of the following conditions
	
	\begin{enumerate}
		\item $B$ is linearly independent.
		\item The only linearly independent subset of $S$ that contains $B$ is $B$ itself.
	\end{enumerate}
\end{definition}

\begin{corollary}
	\hfill\\
	Every vector space has a basis.
\end{corollary}