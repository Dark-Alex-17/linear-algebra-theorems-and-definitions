\section{Vector Spaces}

\begin{definition}
	\hfill\\
	A \textbf{vector space} (or \textbf{linear space}) $V$ over a field $\F$ consists of a set on which two operations (called \textbf{addition} and \textbf{scalar multiplication}, respectively) are defined so that for each pair of elements $x$ and $y$ in $V$ there is a unique element $a$ in $\F$ and each element $x$ in $V$ there is a unique element $ax$ in $V$, such that the following conditions hold:
	
	\begin{description}
		\item[(VS 1)] For all $x, y$ in $V$, $x + y = y + x$ (commutativity of addition).
		\item[(VS 2)] For all $x, y$ in $V$, $(x + y) + z = x + (y + z)$ (associativity of addition).
		\item[(VS 3)] There exists an element in $V$ denoted by $0$ such that $x + 0 = x$ for each $x$ in $V$.
		\item[(VS 4)] For each element $x$ in $V$ there exists an element $y$ in $V$ such that $x + y = 0$.
		\item[(VS 5)] For each element $x$ in $V$, $1x=x$.
		\item[(VS 6)] For each pair of elements $a, b$ in $\F$ and each element $x$ in $V$, $(ab)x = a(bx)$.
		\item[(VS 7)] For each element $a$ in $\F$ and each pair of elements $x, y$ in $V$, $a(x + y) = ax + ay$.
		\item[(VS 8)] For each pair of elements $a, b$ in $\F$ and each element $x$ in $V$, $(a + b)x = ax + bx$.
	\end{description}
	
	The elements $x + y$ and $ax$ are called the \textbf{sum} of $x$ and $y$ and the \textbf{product} of $a$ and $x$, respectively.\\
	
	The elements of the field $\F$ are called \textbf{scalars} and the elements of the vector space $V$ are called \textbf{vectors}.\\
	
	\textbf{Note:} The reader should not confuse this use of the word "vector" with the physical entity discussed in section 1.1: the word "vector" is now being used to describe any element of a vector space.
\end{definition}

\begin{definition}
	\hfill\\
	An object of the form $(a_1, a_2, \dots, a_n)$, where the entries $a_1, a_2, \dots, a_n$ are elements of a field $\F$, is called an \textbf{\textit{n}-tuple} with entries from $\F$. The elements $a_1, a_2, \dots, a_n$ are called the \textbf{entries} or \textbf{components} of the $n$-tuple. Two $n$-tuples $(a_1, a_2, \dots, a_n)$ and $(b_1, b_2, \dots, b_n)$ with entries from a field $\F$ are called \textbf{equal} if $a_i = b_i$ for $i=1, 2, \dots, n$.
\end{definition}

\begin{definition}
	\hfill\\
	Vectors in $\F^n$ may be written as \textbf{column vectors}
	
	\[\begin{pmatrix} a_1 \\ a_2 \\ \vdots \\ a_n \end{pmatrix}\]
	rather than as \textbf{row vectors} $(a_1, a_2, \dots, a_n)$. Since a 1-tuple whose only entry is from $\F$ can be regarded as an element of $\F$, we usually write $\F$ rather than $\F^1$ for the vector space of 1-tuples with entry from $\F$.
\end{definition}

\begin{definition}
	\hfill\\
	An $m \times n$ \textbf{matrix} with entries from a field $\F$ is a rectangular array of the form
	
	\[\begin{pmatrix}
		a_{11} & a_{12} & \dots &a_{1n} \\
		a_{21} & a_{22} & \dots & a_{2n} \\
		\vdots & \vdots & & \vdots \\
		a_{m1} & a_{m2} & \dots & a_{mn}
	\end{pmatrix},\]
	where each entry $a_{ij}\ (1 \leq i \leq m,\ 1 \leq j \leq n)$ is an element of $\F$. We call the entries $a_{ij}$ with $i=j$ the \textbf{diagonal entries} of the matrix. The entries $a_{i1}, a_{i2}, \dots, a_{in}$ compose the \textbf{\textit{i}th row}  of the matrix, and the entries $a_{1j}, a_{2j}, \dots, a_{mj}$ compose the \textbf{\textit{j}th column} of the matrix. The rows of the preceding matrix are regarded as vectors in $\F^n$, and the columns are regarded as vectors in $\F^m$. The $m \times n$ matrix in which each entry equals zero is called the \textbf{zero matrix} and is denoted by $O$.\\
	
	In this book, we denote matrices by capital italic letters (e.g. $A$, $B$, and $C$), and we denote the entry of a matrix $A$ that lies in row $i$ and column $j$ by $A_{ij}$. In addition, if the number of rows and columns of a matrix are equal, the matrix is called \textbf{square}.
	
	Two $m \times n$ matrices $A$ and $B$ are called \textbf{equal} if all their corresponding entries are equal, that is, if $A_{ij} = B_{ij}$ for $1 \leq i \leq m$ and $1 \leq j \leq n$.
\end{definition}

\begin{definition}
	\hfill\\
	The set of all $m \times n$ matrices with entries from a field $\F$ is a vector space which we denote by $M_{m \times n}(\F)$, with the following operations of \textbf{matrix addition} and \textbf{scalar multiplication}: For $A, B \in M_{m \times n}(\F)$ and $c \in \F$,
	
	\[(A + B)_{ij} = A_{ij} + B_{ij}\ \ \ \text{and}\ \ \  (cA)_{ij} = cA_{ij}\]
	for $1 \leq i \leq m$ and $1 \leq j \leq n$.
\end{definition}

\begin{definition}
	\hfill\\
	Let $S$ be any nonempty set and $\F$ be any field, and let $\mathcal{F}(S, \F)$ denote the set of all functions from $S$ to $\F$. Two functions $f$ and $g$ in $\mathcal{F}(S, \F)$ are called \textbf{equal} if $f(s) = g(s)$ for each $s \in S$. The set $\mathcal{F}(S, \F)$ is a vector space with the operations of addition and scalar multiplication defined for $f,g \in \mathcal{F}(S, \F)$ and $c \in \F$ defined by
	
	\[(f + g)(s) = f(s) + g(s)\ \ \ \text{and}\ \ \ (cf)(s) = c[f(s)]\]
	for each $s \in S$. Note that these are the familiar operations of addition and scalar multiplication for functions used in algebra and calculus.
\end{definition}

\begin{definition}
	\hfill\\
	A \textbf{polynomial} with coefficients from a field $\F$ is an expression of the form 
	
	\[f(x)=a_nx^n + a_{n-1}x^{n-1}+\dots+a_1x+a_0,\]
	
	where $n$ is a nonnegative integer and each $a_k$, called the \textbf{coefficient} of $x^k$, is in $\F$. If $f(x)=0$, that is, if $a_n = a_{n-1} = \dots = a_0 = 0$, then $f(x)$ is called the \textbf{zero polynomial} and, for convenience, its degree is defined to be $-1$; otherwise, the \textbf{degree} of a polynomial is defined to be the largest exponent of $x$ that appears in the representation
	
	\[f(x)=a_nx^n + a_{n-1}x^{n-1}+\dots+a_1x+a_0\]
	
	with a nonzero coefficient. Note that the polynomials of degree zero may be written in the form $f(x) = c$ for some nonzero scalar $c$. Two polynomials,
	
	\[f(x)=a_nx^n + a_{n-1}x^{n-1}+\dots+a_1x+a_0\]
	
	and
	
	\[g(x)=b_mx^m + b_{m-1}x^{m-1}+\dots+b_1x+b_0,\]
	
	are called \textbf{equal} if $m=n$ and $a_i = b_i$ for $i=1, 2, \dots, n$.
\end{definition}

\begin{definition}
	\hfill\\
	Let $\F$ be any field. A \textbf{sequence} in $\F$ is a function $\sigma$ from the positive integers into $\F$. In this book, the sequence $\sigma$ such that $\sigma(n) = a_n$ for $n=1, 2, \dots$ is denoted $\{a_n\}$. Let $V$ consist of all sequences $\{a_n\}$ in $\F$ that have only a finite number of nonzero terms $a_n$. If $\{a_n\}$ and $\{b_n\}$ are in $V$ and $t \in \F$, define
	
	\[\{a_n\} + \{b_n\} = \{a_n + b_n\}\ \ \ \text{and}\ \ \ t\{a_n\} = \{ta_n\}\]
\end{definition}

\begin{theorem}[\textbf{Cancellation Law for Vector Addition}]
	\hfill\\
	If $x, y$ and $z$ are vectors in a vector space $V$ such that $x + z = y + z$, then $x = y$.
\end{theorem}

\begin{corollary}
	\hfill\\
	The vector $0$ described in (VS 3) is unique.
\end{corollary}

\begin{corollary}
	\hfill\\
	The vector $y$ described in (VS 4) is unique.
\end{corollary}

\begin{definition}
	\hfill\\
	The vector $0$ in (VS 3) is called the \textbf{zero vector} of $V$, and the vector $y$ in (VS 4) (that is, the unique vector such that $x + y = 0$) is called the \textbf{additive inverse} of $x$ and is denoted by $-x$.
\end{definition}

\begin{theorem}
	\hfill\\
	In any vector space $V$, the following statements are true:
	\begin{enumerate}
		\item $0x = 0$ for each $x \in V$.
		\item $(-a)x = -(ax) = a(-x)$ for each $a \in \F$ and each $x \in V$.
		\item $a0 = 0$ for each $a \in \F$.
	\end{enumerate}
\end{theorem}

\begin{definition}
	\hfill\\
	Let $V=\{0\}$ consist of a single vector $0$ and define $0 + 0 = 0$ and $c0 = 0$ for each scalar $c \in \F$. Then $V$ is called the \textbf{zero vector space}.
\end{definition}

\begin{definition}
	\hfill\\
	A real-valued function $f$ defined on the real line is called an \textbf{even function} if $f(-t) = f(t)$ for each real number $t$, and is called an \textbf{odd function} if $f(-t) = -f(t)$ for each real number $t$.
\end{definition}

\begin{definition}
	\hfill\\
	If $S_1$ and $S_2$ are nonempty subsets of a vector space $V$, then the \textbf{sum} of $S_1$ and $S_2$, denoted $S_1 + S_2$, is the set $\{x + y\ |\ x \in S_1,\ \text{and}\ y \in S_2\}$.
\end{definition}

\begin{definition}
	\hfill\\
	A vector space $V$ is called the \textbf{direct sum} of $W_1$ and $W_2$ if $W_1$ and $W_2$ are subspaces of $V$ such that $W_1 \cap W_2 = \{0\}$ and $W_1 + W_2 = V$. We denote that $V$ is the direct sum of $W_1$ and $W_2$ by writing $V = W_1 \oplus W_2$.
\end{definition}

\begin{definition}
	\hfill\\
	A matrix $M$ is called \textbf{skew-symmetric} if $M^t = -M$.
\end{definition}

\begin{definition}
	\hfill\\
	Let $W$ be a subspace of a vector space $V$ over a field $\F$. For any $v \in V$, the set $\{v\} + W = \{v + w\ |\ w \in W\}$ is called the \textbf{coset of $W$ containing $v$}. It is customary to denote this coset by $v + W$ rather than $\{v\} + W$.
\end{definition}

\begin{definition}
	\hfill\\
	Let $W$ be a subspace of a vector space $V$ over a field $\F$, and let $S := \{v + W\ |\ v \in V\}$ be the set of all cosets of $W$. Then $S$ is called the \textbf{quotient space of $V$ modulo $W$}, and is denoted by $V/W$. Addition and scalar multiplication by the scalars of $\F$ can be defined as follows:
	
	\[(v_1 + W) + (v_2 + W) = (v_1 + v_2) + W\]
	
	for all $v_1, v_2 \in V$, and
	
	\[a(v + W) = av + W\]
	
	for all $v \in V$ and $a \in \F$.
\end{definition}