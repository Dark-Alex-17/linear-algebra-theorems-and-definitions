\section{Linear Combinations and Systems of Linear Equations}

\begin{definition}
	\hfill\\
	Let $V$ be a vector space and $S$ a nonempty subset of $V$. A vector $v \in V$ is called a \textbf{linear combination} of vectors of $S$ if there exist a finite number of vectors $v_1, v_2, \dots, v_n$ in $S$ and scalars $a_1, a_2, \dots, a_n$ in $\F$ such that $v=a_1v_1 + a_2v_2 + \dots + a_nv_n$. In this case we also say that $v$ is a linear combination of $v_1, v_2, \dots, v_n$ and call $a_1, a_2, \dots, a_n$ the \textbf{coefficients} of the linear combination.
\end{definition}

\begin{definition}
	\hfill\\
	Let $S$ be a nonempty subset of a vector space $V$. The \textbf{span} of $S$, denoted $\lspan{S}$, is the set consisting of all linear combinations of the vectors in $S$. For convenience, we define $\lspan{\emptyset} = \{0\}$.
\end{definition}

\begin{theorem}
	\hfill\\
	The span of any subset $S$ of a vector space $V$ is a subspace of $V$. Moreover, any subspace of $V$ that contains $S$ must also contain the span of $S$.
\end{theorem}

\begin{definition}
	\hfill\\
	A subset $S$ of a vector space $V$ \textbf{generates} (or \textbf{spans}) $V$ if $\text{span}(S) = V$. In this case, we also say that the vectors of $S$ generate (or span) $V$.
\end{definition}
